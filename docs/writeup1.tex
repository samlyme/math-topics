\documentclass{article}
\usepackage{graphicx}
\usepackage{todonotes}

\title{Exploratory \& Descriptive Analysis for Math Topics}
\author{Sam Ly \and Nathan Brown \and Jacob Lembach}
\date{\today}

\begin{document}

\maketitle

\section{Dataset Overview}

The dataset used was the Wikipedia Math Essentials dataset. This dataset 
was sourced from Kaggle and represents the relationships between math articles 
on Wikipedia. From the dataset, a directed graph is created with nodes being 
articles and edges being links from on article to another. This yields a graph 
with 1,068 nodes and 27,079 edges. 

This dataset is relevant because studying the relationships between topics 
in mathematics gives us a better understanding of the overall field. We can 
identify understudied topics, find new research areas, etc. Studying this dataset 
also has important educational applications, as it allows educators to make 
more informed course content decisions. Determining which topics to include in 
courses is difficult, and having a quantitative measure of importance can be 
instrumental to making that decision.

\section{Methods Summary}

The dataset was formatted as a large JSON object, so reading and using the 
dataset was relatively easy. The dataset was already cleaned, so no cleaning 
methods were needed. 

The nodes were given an integer ID, which could then be mapped back to the 
article name. Then, the edges were given as a large list of node IDs. Duplicate 
edges, ie. multiple links from and to the same articles, were handled for us,
and were given to us as edge weights. The final graph built must be directed 
because links between articles are directed.

\section{Results and Interpretation}

\subsection{Primary Metrics}

As stated before, our graph contains 1,068 nodes and 27,079 edges in total, 
meaning it tracks 1,068 articles and 27,079 links. However, we notice that our 
density is extremely low, at 0.02. This isn't out of the ordinary, since this is 
not a social network. Intuitively, we know that each individual article tends to 
have a small amount of links relative to every other article out there. 

One notable feature of this graph is that there are no cycles. This can be shown 
when finding the strongly connected components. There are no components with 
size \(> 1\). This means that no two nodes have a ``two-way connection''. This 
also makes sense, as a chain of articles linking back on itself isn't very useful. 
This also means that the graph is \textbf{acyclic}, thus there exists valid 
topological sorts of this graph. Although not used, the topological sort of 
the graph may provide valuable insights. 

\subsection{Communities and Clustering}

In theory, finding the communities of nodes within our graph would give us 
insight on the relationships between the branches of mathematics. 

\includegraphics[width=0.8\textwidth]{coommunities.png}

However in practice, this graph tells us more about how Wikipedia has ended up 
structuring the articles for math, rather than the underlying structure of 
math knowledge. Further experimentation is needed to find the information we 
are looking for. 

By running a greedy modularity search, we find 5 primary communities. We define 
the hub of each community to be the node with the highest out degree. We will 
see later that the degree distribution of the out degree follows that of a 
scale-free graph, so the degree centrality is actually a viable measure.

We find the hubs and their corresponding out degree to be:
\begin{verbatim}
Mathematics 533
Real number 290
Calculus 197
Statistics 160
Prisoner's dilemma 19
\end{verbatim}

Here we see some issues with our dataset. Extremely broad articles like ``Mathematics''
have isn't really a ``math topic'' per se, and can get in the way of actually 
useful information. Also, definition-heavily articles like ``Real number'' end 
up seeming like important topics because of their high out degree, despite 
not necessarily meetin the criteria for an actual topic. The ``Calculus'' and 
``Statistics'' nodes make the most sense. 

An artifact of our clustering algorithm is the ``Prisoner's dilemma'' being 
falsely considered a hub. When looking at the other nodes in its community, we 
see that the actual hub should be the article ``Game theory'', however it was 
somehow lost when we computed the communities.

\subsection{Degree Distribution}

First, because our graph is directed, there are two distict degree distributions:
the in degree and out degree.

The in degree distribution seems to follow a Poisson distribution, while the out 
degree distribution follows the degree distribution of a scale-free network. 

\includegraphics[width=0.8\textwidth]{in_degree_dist.png}

\includegraphics[width=0.8\textwidth]{out_degree_dist.png}

This is more clearly shown on a log-log plot.

\includegraphics[width=0.8\textwidth]{log_log_degree_dist.png}

This pattern emerges because articles about broad topics tend to link outwards to 
many other articles. At the same time, there are fewer broad topics out there to 
write articles on. So the scale-free distribution emerges as broad articles 
are ``preferred'' to link outwards. 

Counterintuitively, inwards links seem to be more random, hence the Poisson 
distribution. Although we may falsely assume that broad articles like ``Mathematics''
should also be preferred as link targets, in reality, articles for broad topics 
don't get linked to any more than other articles. An obvious example of this would 
be that literally every node in the dataset semantically relates to the ``Mathematics''
article, but it doesn't mean that every single article written needs to refer
to ``Mathematics''. Another way to view this is from the perspective of the writer. 
When referring to other articles, the usefulness of said article is independent 
from its ``broadness''. Thus, the likelihood of an article receiving a new 
inwards link is practically random. 

\section{Visualization Discussion}
Looking at the figures, the visuals helped us understand how the Wikipedia Math network is structured. The community graph on page 2 shows five main clusters that represent different areas of mathematics. The color groups make it easy to see which articles are closely related. The “Mathematics” node sits near the center and connects many clusters, showing how broad topics link the rest of the network. It is apparent that broad articles dominate as hubs, which can hide more meaningful relationships between real math topics.
The degree distribution plots show two main patterns. The in-degree distribution looks roughly like a Poisson curve, meaning most articles get a random, moderate number of incoming links. The out-degree distribution is very uneven with most articles linked to only a few others, while a few nodes link to hundreds. On the log-log plot, this creates a straight-line shape typical of a scale-free network. This means that a few hub articles, like “Mathematics” or “Calculus,” connect to many others, while most topics stay fairly isolated.
Together, these visuals show that the Wikipedia math network is uneven and hierarchical. Some articles act as major connectors across topics, while most form smaller, tighter groups. Seeing these patterns helps explain how math knowledge is linked and organized online.

\section{Reflection and Next Steps}
From our analysis, several new questions and hypotheses emerged, such as how we 
can quantify article ``broadness.'' For future analyses, we plan to explore 
measures of centrality, diffusion, and community detection. Since the dataset 
includes time series data, it would be useful to test whether a diffusion model 
appropriately captures the dynamics present. Additionally, we intend to experiment 
with other community detection algorithms to improve the robustness of the results. 
One technical challenge we encountered was that identifying communities in articles 
can become unexpectedly complex and “weird,” often producing results that are 
difficult to interpret.

\end{document}
